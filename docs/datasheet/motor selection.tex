\chapter{Motor selection considerations}
One non-obvious matter that differs FOC-enabled motor controllers (like Zubax Komar)
from conventional controllers with trapezoidal or six step commutation 
is so called voltage utilization factor.
Every BLDC motor has a characteristic that defines its theoretical maximum rotational speed. 
It is called  motor speed constant K\textsubscript{v}. 
It is measured in revolutions per minute (RPM) per volt or radians per volt second [rad/(V*s)].
Physical meaning of this constant is the number of revolutions per minute (rpm) that a motor turns when 1V (one volt) 
is applied with no load attached to that motor. 

FOC-enabled motor controllers have additional voltage utilization factor that decreases the maximum RPM 
for a given motor and supply voltage. For Komar this factor is:

\[F\textsubscript{util} = \frac{0.91}{\sqrt{3}}\]
\[RPM\textsubscript{max} = K\textsubscript{v} \times V\textsubscript{supply} \times F\textsubscript{util}\]

RPM\textsubscript{max} of a FOC enabled motor controller will always be lower than RPM\textsubscript{max} 
of a conventional controller with trapezoidal or six step commutation. This should be taken into account 
when designing a propulsion system.

For example, a motor with speed constant K\textsubscript{v} = 320 controlled by Komar running on fully charged 10S $\text{LiCoO}_\text{2}$ battery will have the following theoretical maximum RPM:

\[RPM\textsubscript{max} = 320 \times 10 \times 4.2 \times \frac{0.91}{\sqrt{3}} = 7061\]